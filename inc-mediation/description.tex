\documentclass[10pt]{article}
\usepackage[margin=1in,top=1.75in,bottom=1in,headheight=85pt]{geometry}

% font
\usepackage{tgheros}
\renewcommand*\familydefault{\sfdefault}
\usepackage[T1]{fontenc}

\usepackage{letltxmacro}
\LetLtxMacro\oldttfamily\ttfamily
\DeclareRobustCommand{\ttfamily}{\oldttfamily\csname ttsize\endcsname}
\newcommand{\setttsize}[1]{\def\ttsize{#1}}%

% line-spacing
\renewcommand{\baselinestretch}{1.1}

\usepackage{amsmath,amsthm,amssymb,graphicx,array}
\usepackage{tabularx}
\usepackage{verbatim,listings}
\usepackage{mdframed}
\usepackage{fancyvrb}
\usepackage{blindtext}

% header and footer
\usepackage{fancyhdr}
\pagestyle{fancy}
\chead{\includegraphics[width=0.75\paperwidth,height=66pt]{header-inc.jpg}}
\lfoot{ICPC INC 2025}
\renewcommand{\headrulewidth}{0.75pt}
\renewcommand{\footrulewidth}{0.75pt}

% no page number
\pagenumbering{gobble}

% no indent on new paragraph
\setlength\parindent{0mm}
\setlength{\parskip}{2mm}

% commands
\newcommand{\ptitle}[2]{%
\rfoot{Problem #1. #2}
\section*{\centering {Problem #1}\\\vspace{2mm}{\LARGE #2}\vspace{5mm}}}

\newcommand{\psection}[1]{\vspace{-3mm}\subsubsection*{#1}\vspace{-3mm}}

\newcommand{\blankpage}{\newpage\vspace*{\fill}{\centering {\em This page is intentionally left blank.}\par}\vspace*{\fill}}

% Define the \exmp command that works within example environment
\newcommand{\exmp}[3]{%
    \vspace{2mm}
    \begin{tabular}{|p{0.47\textwidth}|p{0.47\textwidth}|}
    \multicolumn{1}{l}{\textbf{Sample Input #1}} & \multicolumn{1}{l}{\textbf{Sample Output #1}} \\
    \hline
    \begin{minipage}[t]{0.47\textwidth}
    \ttfamily\small
    #2
    \end{minipage}
    \vspace{1pt}
    &
    \begin{minipage}[t]{0.47\textwidth}
    \ttfamily\small
    #3
    \end{minipage}
    \vspace{1pt}
    \\
    \hline
    \end{tabular}
    \vspace{2mm}
}

%-------------------------------------------------------------------------------
\usepackage[table,xcdraw]{xcolor}
\usepackage{float}

\begin{document}

\ptitle{J}{Mediation}

You are the mayor of a tree-structured city with $N$ districts, numbered from $1$ to $N$, connected by $N - 1$ roads, numbered from $1$ to $N - 1$.
Road $i$ connects district $U_i$ and district $V_i$ bidirectionally with weight $W_i$.
Two districts $S_1$ and $S_2$ have been marked as \textbf{mediator districts}.
The travel cost between district $x$ and district $y$, denoted by $d(x, y)$, is the minimum sum of weights of the roads you need to pass through.

Whenever a conflict arise between any two districts, the mediator districts are required to travel to the conflicting districts.
The \textbf{mediation cost} for two conflicting districts $u$ and $v$, denoted by $M(u, v)$, is the maximum travel cost of the mediator districts to the nearest conflicting district.
Formally, $M(x, y)$ can be calculated as follows.

\[M(u, v) = \max(\min(d(u, S_1), d(v, S_1)), \min(d(u, S_2), d(v, S_2)))\]

Calculate the sum of mediation cost $M(u, v)$ over all $1 \leq u < v \leq N$.


\psection{Input}
The first line contains three integers: $N$, $S_1$, and $S_2$ ($2 \le N \le 200\,000$; $1 \leq S_1 < S_2 \leq N$).

The next $N-1$ lines contains integers $U_i$, $V_i$, and $W_i$ ($1 \le U_i < V_i \le N$; $1 \leq W_i \leq 100$) describing an edge.


\psection{Output}
Output the sum of mediation cost in a single line.

\exmp{1}{
6 2 4\\
1 4 5\\
3 4 1\\
4 5 1\\
2 5 2\\
2 6 2
}{
35
}

\textit{Explanation of Sample 1:} The city is illustrated as follows.

\includegraphics[width=0.3\textwidth]{render/image.png}

The values of $M(u, v)$ over all $1 \leq u < v \leq N$ are presented as follows.

\begin{table}[H]
\begin{tabular}{l|llllll}
\textbf{u \textbackslash{} v} & \textbf{1}               & \textbf{2}               & \textbf{3}               & \textbf{4}               & \textbf{5}               & \textbf{6} \\ \hline
\textbf{1}                  & \cellcolor[HTML]{EFEFEF} & 3                        & 4                        & 3                        & 2                        & 5          \\
\textbf{2}                  & \cellcolor[HTML]{EFEFEF} & \cellcolor[HTML]{EFEFEF} & 1                        & 0                        & 1                        & 3          \\
\textbf{3}                  & \cellcolor[HTML]{EFEFEF} & \cellcolor[HTML]{EFEFEF} & \cellcolor[HTML]{EFEFEF} & 3                        & 2                        & 2          \\
\textbf{4}                  & \cellcolor[HTML]{EFEFEF} & \cellcolor[HTML]{EFEFEF} & \cellcolor[HTML]{EFEFEF} & \cellcolor[HTML]{EFEFEF} & 2                        & 2          \\
\textbf{5}                  & \cellcolor[HTML]{EFEFEF} & \cellcolor[HTML]{EFEFEF} & \cellcolor[HTML]{EFEFEF} & \cellcolor[HTML]{EFEFEF} & \cellcolor[HTML]{EFEFEF} & 2         
\end{tabular}
\end{table}

The sum of all $M(u, v)$ is 35.

\end{document}