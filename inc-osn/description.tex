\documentclass[10pt]{article}
\usepackage[margin=1in,top=1.75in,bottom=1in,headheight=85pt]{geometry}

% font
\usepackage{tgheros}
\renewcommand*\familydefault{\sfdefault}
\usepackage[T1]{fontenc}

\usepackage{letltxmacro}
\LetLtxMacro\oldttfamily\ttfamily
\DeclareRobustCommand{\ttfamily}{\oldttfamily\csname ttsize\endcsname}
\newcommand{\setttsize}[1]{\def\ttsize{#1}}%

% line-spacing
\renewcommand{\baselinestretch}{1.1}

\usepackage{amsmath,amsthm,amssymb,graphicx,array}
\usepackage{tabularx}
\usepackage{verbatim,listings}
\usepackage{mdframed}
\usepackage{fancyvrb}
\usepackage{blindtext}

% header and footer
\usepackage{fancyhdr}
\pagestyle{fancy}
\chead{\includegraphics[width=0.75\paperwidth,height=66pt]{header-inc.jpg}}
\lfoot{ICPC INC 2025}
\renewcommand{\headrulewidth}{0.75pt}
\renewcommand{\footrulewidth}{0.75pt}

% no page number
\pagenumbering{gobble}

% no indent on new paragraph
\setlength\parindent{0mm}
\setlength{\parskip}{2mm}

% commands
\newcommand{\ptitle}[2]{%
\rfoot{Problem #1. #2}
\section*{\centering {Problem #1}\\\vspace{2mm}{\LARGE #2}\vspace{5mm}}}

\newcommand{\psection}[1]{\vspace{-3mm}\subsubsection*{#1}\vspace{-3mm}}

\newcommand{\blankpage}{\newpage\vspace*{\fill}{\centering {\em This page is intentionally left blank.}\par}\vspace*{\fill}}

% Define the \exmp command that works within example environment
\newcommand{\exmp}[3]{%
    \vspace{2mm}
    \begin{tabular}{|p{0.47\textwidth}|p{0.47\textwidth}|}
    \multicolumn{1}{l}{\textbf{Sample Input #1}} & \multicolumn{1}{l}{\textbf{Sample Output #1}} \\
    \hline
    \begin{minipage}[t]{0.47\textwidth}
    \ttfamily\small
    #2
    \end{minipage}
    \vspace{1pt}
    &
    \begin{minipage}[t]{0.47\textwidth}
    \ttfamily\small
    #3
    \end{minipage}
    \vspace{1pt}
    \\
    \hline
    \end{tabular}
    \vspace{2mm}
}

%-------------------------------------------------------------------------------

\begin{document}

\ptitle{A}{National Science Olympiad}

In the recent National Science Olympiad, a new tie-breaker rule was implemented using time penalties measured in minutes.

The olympiad ran for two days and participated in by $N$ contestants.
The $i$-th participant, whose name is $S_i$, received $A_i$ points with time penalty $B_i$ on the first day, and received $C_i$ points with time penalty $D_i$ on the second day.

The contestants are ranked based on their total points from both days, with \textbf{higher points} being ranked higher.
Contestants who receive the same total points will be ranked based on the sum of their time penalties from both days, with \textbf{lower penalties} being ranked higher.
If there are still ties, they will be ranked based on their name, with the \textbf{lexicographically smaller name} being ranked higher.

Your task is to output the name of the contestants ranked from highest to lowest.

\psection{Input}

The first line contains an integer $N$ ($1 \le N \le 100$).
Each of the next $N$ lines contains a string $S_i$ ($1 \leq |S_i| \leq 10$), containing uppercase English alphabets, followed by four integers $A_i$, $B_i$, $C_i$, and $D_i$ ($0 \le A_i, B_i, C_i, D_i \le 300$) representing their points and penalties from both days.

\psection{Output}

Output $N$ lines, each containing the name of the contestants, in order from the higher-ranked to the lower-ranked.

\exmp{1}{
5\\
ANDI 200 120 150 130\\
BUDI 170 70 180 170\\
CUPU 0 300 0 300\\
DEWA 300 0 300 0\\
MALANG 0 300 0 300
}{
DEWA\\
BUDI\\
ANDI\\
CUPU\\
MALANG
}

\textit{Explanation of Sample 1:} Let us consider the ranks between ANDI and BUDI.
Both ANDI and BUDI received the same total points, which is $350$, but BUDI is ranked higher because BUDI's time penalty (which is $240$) is lower than ANDI's (which is $250$).

\blankpage % use this if problem is only 1 page

\end{document}
