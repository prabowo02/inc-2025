\documentclass[10pt]{article}
\usepackage[margin=1in,top=1.75in,bottom=1in,headheight=85pt]{geometry}

% font
\usepackage{tgheros}
\renewcommand*\familydefault{\sfdefault}
\usepackage[T1]{fontenc}

\usepackage{letltxmacro}
\LetLtxMacro\oldttfamily\ttfamily
\DeclareRobustCommand{\ttfamily}{\oldttfamily\csname ttsize\endcsname}
\newcommand{\setttsize}[1]{\def\ttsize{#1}}%

% line-spacing
\renewcommand{\baselinestretch}{1.1}

\usepackage{amsmath,amsthm,amssymb,graphicx,array}
\usepackage{tabularx}
\usepackage{verbatim,listings}
\usepackage{mdframed}
\usepackage{fancyvrb}
\usepackage{blindtext}

% header and footer
\usepackage{fancyhdr}
\pagestyle{fancy}
\chead{\includegraphics[width=0.75\paperwidth,height=66pt]{header-inc.jpg}}
\lfoot{ICPC INC 2025}
\renewcommand{\headrulewidth}{0.75pt}
\renewcommand{\footrulewidth}{0.75pt}

% no page number
\pagenumbering{gobble}

% no indent on new paragraph
\setlength\parindent{0mm}
\setlength{\parskip}{2mm}

% commands
\newcommand{\ptitle}[2]{%
\rfoot{Problem #1. #2}
\section*{\centering {Problem #1}\\\vspace{2mm}{\LARGE #2}\vspace{5mm}}}

\newcommand{\psection}[1]{\vspace{-3mm}\subsubsection*{#1}\vspace{-3mm}}

\newcommand{\blankpage}{\newpage\vspace*{\fill}{\centering {\em This page is intentionally left blank.}\par}\vspace*{\fill}}

% Define the \exmp command that works within example environment
\newcommand{\exmp}[3]{%
    \vspace{2mm}
    \begin{tabular}{|p{0.47\textwidth}|p{0.47\textwidth}|}
    \multicolumn{1}{l}{\textbf{Sample Input #1}} & \multicolumn{1}{l}{\textbf{Sample Output #1}} \\
    \hline
    \begin{minipage}[t]{0.47\textwidth}
    \ttfamily\small
    #2
    \end{minipage}
    \vspace{1pt}
    &
    \begin{minipage}[t]{0.47\textwidth}
    \ttfamily\small
    #3
    \end{minipage}
    \vspace{1pt}
    \\
    \hline
    \end{tabular}
    \vspace{2mm}
}

%-------------------------------------------------------------------------------

\begin{document}

\ptitle{E}{Xordition Robot}

You have a robot that contains $N$ modules, numbered from $1$ to $N$.
Each module accepts an integer and outputs an integer.
The output of module $i$ becomes the input of module $i+1$ (for $1 \le i \le N - 1$).

The specification of module $i$ is either:

\begin{itemize}
  \item \texttt{+ k}: given an integer $x$ ($0 \leq x < 16$), the module outputs $(x + k) \bmod 16$; or
  \item \texttt{x k}: given an integer $x$ ($0 \leq x < 16$), the module outputs $x \oplus k$, where $\oplus$ represents the bitwise XOR operator.
\end{itemize}

There are $Q$ replacements, and the $j$-th is of the form:

\begin{itemize}
  \item \texttt{i t k}: replace module $i$ to a module with specification \texttt{t k}, where $t$ is either \texttt{+} or \texttt{x}.
\end{itemize}

Each time a replacement is done, find the output of module $N$ when module $1$ is given an input $0$.

\psection{Input}

The first line contains two integers $N$ and $Q$ ($1 \leq N, Q \leq 200\;000$).
Each of the next $N$ lines contains a character of either \texttt{+} or \texttt{x} followed by an integer $k$ ($0 \leq k < 16$) representing the module.

The next $Q$ lines contains an integer $i$ ($1 \leq i \leq N$), followed by a character \texttt{+} or \texttt{x}, and finally an integer $k$ ($0 \leq k < 16$), meaning that you have to replace module $i$ to the specified module.

\psection{Output}

Output $Q$ lines, each containing the output of module $N$, after each replacement, when given an input $0$ to module $1$.

\exmp{1}{
4 2\\
+ 3\\
x 5\\
x 9\\
+ 15\\
2 + 8\\
1 x 10
}{
1\\
10
}

\textit{Explanation of Sample 1:} After the first replacement, the modules are: \texttt{+ 3}, \texttt{+ 8}, \texttt{x 9}, \texttt{+ 15}

The output of module $N$ is then $(((0 + 3) + 8) \oplus 9) + 15)$ is $1$.

After the second replacement, the modules are: \texttt{x 10}, \texttt{+ 8}, \texttt{x 9}, \texttt{+ 15}

The output of module $N$ is then $(((0 \oplus 10) + 8) \oplus 9) + 15)$ is $10$.

\blankpage % use this if problem is only 1 page

\end{document}
