\documentclass[10pt]{article}
\usepackage[margin=1in,top=1.75in,bottom=1in,headheight=85pt]{geometry}

% font
\usepackage{tgheros}
\renewcommand*\familydefault{\sfdefault}
\usepackage[T1]{fontenc}

\usepackage{letltxmacro}
\LetLtxMacro\oldttfamily\ttfamily
\DeclareRobustCommand{\ttfamily}{\oldttfamily\csname ttsize\endcsname}
\newcommand{\setttsize}[1]{\def\ttsize{#1}}%

% line-spacing
\renewcommand{\baselinestretch}{1.1}

\usepackage{amsmath,amsthm,amssymb,graphicx,array}
\usepackage{tabularx}
\usepackage{verbatim,listings}
\usepackage{mdframed}
\usepackage{fancyvrb}
\usepackage{blindtext}

% header and footer
\usepackage{fancyhdr}
\pagestyle{fancy}
\chead{\includegraphics[width=0.75\paperwidth,height=66pt]{header-inc.jpg}}
\lfoot{ICPC INC 2025}
\renewcommand{\headrulewidth}{0.75pt}
\renewcommand{\footrulewidth}{0.75pt}

% no page number
\pagenumbering{gobble}

% no indent on new paragraph
\setlength\parindent{0mm}
\setlength{\parskip}{2mm}

% commands
\newcommand{\ptitle}[2]{%
\rfoot{Problem #1. #2}
\section*{\centering {Problem #1}\\\vspace{2mm}{\LARGE #2}\vspace{5mm}}}

\newcommand{\psection}[1]{\vspace{-3mm}\subsubsection*{#1}\vspace{-3mm}}

\newcommand{\blankpage}{\newpage\vspace*{\fill}{\centering {\em This page is intentionally left blank.}\par}\vspace*{\fill}}

% Define the \exmp command that works within example environment
\newcommand{\exmp}[3]{%
    \vspace{2mm}
    \begin{tabular}{|p{0.47\textwidth}|p{0.47\textwidth}|}
    \multicolumn{1}{l}{\textbf{Sample Input #1}} & \multicolumn{1}{l}{\textbf{Sample Output #1}} \\
    \hline
    \begin{minipage}[t]{0.47\textwidth}
    \ttfamily\small
    #2
    \end{minipage}
    \vspace{1pt}
    &
    \begin{minipage}[t]{0.47\textwidth}
    \ttfamily\small
    #3
    \end{minipage}
    \vspace{1pt}
    \\
    \hline
    \end{tabular}
    \vspace{2mm}
}

%-------------------------------------------------------------------------------

\begin{document}

\ptitle{G}{Mex XOR}

You initially have an empty set $S$, and an integer $K$.
You will then have to process $Q$ queries, each giving you an integer $X$, meaning that you will have to \textbf{insert} $X$ into $S$ if $X \notin S$, or \textbf{remove} $X$ from $S$ if $X \in S$.

After each query, you would like to know the following.
Find the minimum of $\text{MEX}(\{s \oplus i : s \in S \})$ for all $0 \le i \le K$.

The operator $\oplus$ is the bitwise XOR operation, while $\text{MEX}$ is a function that returns the smallest non-negative integer that does not appear in the set. In particular, the $\text{MEX}$ of an empty set is $0$.

\psection{Input}

The first line contains two integer $Q$ and $K$ ($1 \le Q \le 200\;000$; $0 \le K < 2^{30}$).

Each of the next $Q$ lines contains an integer $X$ ($0 \le X < 2^{30}$).

\psection{Output}

Output $Q$ lines, representing the minimum $\text{MEX}$ value after each query. 

\exmp{1}{
4 2\\
1\\
0\\
2\\
1
}{
0\\
0\\
1\\
0
}

\textit{Explanation of Sample 1:} After the first query, the set $S$ is $\{1\}$. We can see that $\text{MEX}(\{1 \oplus 0\}) = 0$, and this is the minimum possible value.

After the third query, the set $S$ is $\{0, 1, 2\}$. The values to consider are as follows:

\begin{itemize}
  \item $\text{MEX}(\{0 \oplus 0, 1 \oplus 0, 2 \oplus 0\}) = \text{MEX}(\{0, 1, 2\}) = 3$.
  \item $\text{MEX}(\{0 \oplus 1, 1 \oplus 1, 2 \oplus 1\}) = \text{MEX}(\{1, 0, 3\}) = 2$.
  \item $\text{MEX}(\{0 \oplus 2, 1 \oplus 2, 2 \oplus 2\}) = \text{MEX}(\{2, 3, 0\}) = 1$.
\end{itemize}

The minimum among them is $1$.

After the fourth query, the set $S$ is $\{0, 2\}$ and $\text{MEX}(\{0 \oplus 1, 2 \oplus 1\}) = 0$.

\blankpage % use this if problem is only 1 page

\end{document}