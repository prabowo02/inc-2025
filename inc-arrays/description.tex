\documentclass[10pt]{article}
\usepackage[margin=1in,top=1.75in,bottom=1in,headheight=85pt]{geometry}

% font
\usepackage{tgheros}
\renewcommand*\familydefault{\sfdefault}
\usepackage[T1]{fontenc}

\usepackage{letltxmacro}
\LetLtxMacro\oldttfamily\ttfamily
\DeclareRobustCommand{\ttfamily}{\oldttfamily\csname ttsize\endcsname}
\newcommand{\setttsize}[1]{\def\ttsize{#1}}%

% line-spacing
\renewcommand{\baselinestretch}{1.1}

\usepackage{amsmath,amsthm,amssymb,graphicx,array}
\usepackage{tabularx}
\usepackage{verbatim,listings}
\usepackage{mdframed}
\usepackage{fancyvrb}
\usepackage{blindtext}

% header and footer
\usepackage{fancyhdr}
\pagestyle{fancy}
\chead{\includegraphics[width=0.75\paperwidth,height=66pt]{header-inc.jpg}}
\lfoot{ICPC INC 2025}
\renewcommand{\headrulewidth}{0.75pt}
\renewcommand{\footrulewidth}{0.75pt}

% no page number
\pagenumbering{gobble}

% no indent on new paragraph
\setlength\parindent{0mm}
\setlength{\parskip}{2mm}

% commands
\newcommand{\ptitle}[2]{%
\rfoot{Problem #1. #2}
\section*{\centering {Problem #1}\\\vspace{2mm}{\LARGE #2}\vspace{5mm}}}

\newcommand{\psection}[1]{\vspace{-3mm}\subsubsection*{#1}\vspace{-3mm}}

\newcommand{\blankpage}{\newpage\vspace*{\fill}{\centering {\em This page is intentionally left blank.}\par}\vspace*{\fill}}

% Define the \exmp command that works within example environment
\newcommand{\exmp}[3]{%
    \vspace{2mm}
    \begin{tabular}{|p{0.47\textwidth}|p{0.47\textwidth}|}
    \multicolumn{1}{l}{\textbf{Sample Input #1}} & \multicolumn{1}{l}{\textbf{Sample Output #1}} \\
    \hline
    \begin{minipage}[t]{0.47\textwidth}
    \ttfamily\small
    #2
    \end{minipage}
    \vspace{1pt}
    &
    \begin{minipage}[t]{0.47\textwidth}
    \ttfamily\small
    #3
    \end{minipage}
    \vspace{1pt}
    \\
    \hline
    \end{tabular}
    \vspace{2mm}
}

%-------------------------------------------------------------------------------

\begin{document}

\ptitle{L}{Honourable Arrays}

You are given a set $S$ containing $N$ different positive integers.
You are also given a positive integer $K$ and a prime number $M$.

An array is said to be \textit{honourable} if each of its elements is in $S$, and the product of all elements in the array is $K$ modulo $M$.

For a given integer $L$, count the number of different honourable arrays with length $L$.
Two arrays of length $L$ are said to be different if there exists an index such that the elements in both arrays differ at that index.
Output the count modulo $998\;244\;353$.

\psection{Input}

The first line contains four integers $N$, $K$, $M$, and $L$ ($1 \leq N, K < M$; $2 \leq M \leq 100\;000$; $1 \leq L \leq 10^9$; $M$ is a prime).
The second line contains $N$ different integers representing $S$, each is a positive integer less than $M$.

\psection{Output}

Output an integer representing the number of different honourable arrays modulo $998\;244\;353$.

\exmp{1}{
2 1 3 4\\
1 2
}{
8
}

\textit{Explanation of Sample 1:} the different honourable arrays of size $4$ are: $[1, 1, 1, 1]$, $[1, 1, 2, 2]$, $[1, 2, 1, 2]$, $[1, 2, 2, 1]$, $[2, 1, 1, 2]$, $[2, 1, 2, 1]$, $[2, 2, 1, 1]$, $[2, 2, 2, 2]$.

\exmp{2}{
2 1 3 1000\\
1 2
}{
510735315
}


\blankpage % use this if problem is only 1 page

\end{document}
