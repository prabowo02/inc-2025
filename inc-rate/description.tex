\documentclass[10pt]{article}
\usepackage[margin=1in,top=1.75in,bottom=1in,headheight=85pt]{geometry}

% font
\usepackage{tgheros}
\renewcommand*\familydefault{\sfdefault}
\usepackage[T1]{fontenc}

\usepackage{letltxmacro}
\LetLtxMacro\oldttfamily\ttfamily
\DeclareRobustCommand{\ttfamily}{\oldttfamily\csname ttsize\endcsname}
\newcommand{\setttsize}[1]{\def\ttsize{#1}}%

% line-spacing
\renewcommand{\baselinestretch}{1.1}

\usepackage{amsmath,amsthm,amssymb,graphicx,array}
\usepackage{tabularx}
\usepackage{verbatim,listings}
\usepackage{mdframed}
\usepackage{fancyvrb}
\usepackage{blindtext}

% header and footer
\usepackage{fancyhdr}
\pagestyle{fancy}
\chead{\includegraphics[width=0.75\paperwidth,height=66pt]{header-inc.jpg}}
\lfoot{ICPC INC 2025}
\renewcommand{\headrulewidth}{0.75pt}
\renewcommand{\footrulewidth}{0.75pt}

% no page number
\pagenumbering{gobble}

% no indent on new paragraph
\setlength\parindent{0mm}
\setlength{\parskip}{2mm}

% commands
\newcommand{\ptitle}[2]{%
\rfoot{Problem #1. #2}
\section*{\centering {Problem #1}\\\vspace{2mm}{\LARGE #2}\vspace{5mm}}}

\newcommand{\psection}[1]{\vspace{-3mm}\subsubsection*{#1}\vspace{-3mm}}

\newcommand{\blankpage}{\newpage\vspace*{\fill}{\centering {\em This page is intentionally left blank.}\par}\vspace*{\fill}}

% Define the \exmp command that works within example environment
\newcommand{\exmp}[3]{%
    \vspace{2mm}
    \begin{tabular}{|p{0.47\textwidth}|p{0.47\textwidth}|}
    \multicolumn{1}{l}{\textbf{Sample Input #1}} & \multicolumn{1}{l}{\textbf{Sample Output #1}} \\
    \hline
    \begin{minipage}[t]{0.47\textwidth}
    \ttfamily\small
    #2
    \end{minipage}
    \vspace{1pt}
    &
    \begin{minipage}[t]{0.47\textwidth}
    \ttfamily\small
    #3
    \end{minipage}
    \vspace{1pt}
    \\
    \hline
    \end{tabular}
    \vspace{2mm}
}

%-------------------------------------------------------------------------------

\begin{document}

\ptitle{F}{Food Rating}

In the new food delivery app \textit{Touch}, customers are able to rate the driver with integer scores from $L$ to $R$ inclusive.
Drivers will get a bonus based on how high their average rating is.
However, some drivers may abuse this system.
A driver delivering food to $1$ customer may get $5.0$ average rating, while another driver delivering food to $5$ customers may get $4.8$ average rating.

As the owner of the app, you need to ensure fairness in the bonus system.
To do that, you need to know: for a driver to have an average rating of exactly $X$, what is the minimum number of delivery $k$, such that there exists a scenario where the average rating given by $k$ customers is exactly $X$.
In addition to that, output any list of $k$ integers within $L$ to $R$ such that the average of the list is exactly $X$.

\psection{Input}

The first line contains a real number $X$ ($0 \le X \le 1000$).
The number $X$ contains at most $6$ digits, including both digits before and after the decimal separator (if any).

The second line contains two integers $L$ and $R$ ($1 \le L \le R \le 1000$).

\psection{Output}

If there exists a scenario where a driver can get an average rating exactly $X$, output in the first line, the minimum integer $k$ representing the minimum number of customers giving the rating. In the next line, output $k$ integers between $L$ and $R$ representing the rating given by the customers.

If there is no such scenario, output $-1$ in a single line. 

\exmp{1}{
8.6\\
1 10
}{
5\\
10 9 10 7 7
}

\textit{Explanation of Sample 1:} The average of $[10, 9, 10, 7, 7]$ is exactly $8.6$.
It can be proven such that there is no valid list with four or less integers.

\exmp{2}{
9\\
1 10
}{
1\\
9
}

\exmp{3}{
2.79\\
3 5
}{
-1
}

\blankpage % use this if problem is only 1 page

\end{document}