\documentclass[10pt]{article}
\usepackage[margin=1in,top=1.75in,bottom=1in,headheight=85pt]{geometry}

% font
\usepackage{tgheros}
\renewcommand*\familydefault{\sfdefault}
\usepackage[T1]{fontenc}

\usepackage{letltxmacro}
\LetLtxMacro\oldttfamily\ttfamily
\DeclareRobustCommand{\ttfamily}{\oldttfamily\csname ttsize\endcsname}
\newcommand{\setttsize}[1]{\def\ttsize{#1}}%

% line-spacing
\renewcommand{\baselinestretch}{1.1}

\usepackage{amsmath,amsthm,amssymb,graphicx,array}
\usepackage{tabularx}
\usepackage{verbatim,listings}
\usepackage{mdframed}
\usepackage{fancyvrb}
\usepackage{blindtext}

% header and footer
\usepackage{fancyhdr}
\pagestyle{fancy}
\chead{\includegraphics[width=0.75\paperwidth,height=66pt]{header-inc.jpg}}
\lfoot{ICPC INC 2025}
\renewcommand{\headrulewidth}{0.75pt}
\renewcommand{\footrulewidth}{0.75pt}

% no page number
\pagenumbering{gobble}

% no indent on new paragraph
\setlength\parindent{0mm}
\setlength{\parskip}{2mm}

% commands
\newcommand{\ptitle}[2]{%
\rfoot{Problem #1. #2}
\section*{\centering {Problem #1}\\\vspace{2mm}{\LARGE #2}\vspace{5mm}}}

\newcommand{\psection}[1]{\vspace{-3mm}\subsubsection*{#1}\vspace{-3mm}}

\newcommand{\blankpage}{\newpage\vspace*{\fill}{\centering {\em This page is intentionally left blank.}\par}\vspace*{\fill}}

% Define the \exmp command that works within example environment
\newcommand{\exmp}[3]{%
    \vspace{2mm}
    \begin{tabular}{|p{0.47\textwidth}|p{0.47\textwidth}|}
    \multicolumn{1}{l}{\textbf{Sample Input #1}} & \multicolumn{1}{l}{\textbf{Sample Output #1}} \\
    \hline
    \begin{minipage}[t]{0.47\textwidth}
    \ttfamily\small
    #2
    \end{minipage}
    \vspace{1pt}
    &
    \begin{minipage}[t]{0.47\textwidth}
    \ttfamily\small
    #3
    \end{minipage}
    \vspace{1pt}
    \\
    \hline
    \end{tabular}
    \vspace{2mm}
}

%-------------------------------------------------------------------------------

\begin{document}

\ptitle{B}{Construct BFS Graph}

You are currently researching a graph traversal algorithm called the Breadth First Search (BFS).
Suppose there is a graph of $N$ nodes, numbered from $1$ to $N$, and an adjacency matrix $A$, for which node $u$ can traverse to node $v$ if $A_{u, v}$ is $1$, otherwise it is $0$.
The following pseudocode will output the order the nodes that are visited in a BFS algorithm.

\vspace{-3mm}
\begin{verbatim}
    BFS(A[1..N][1..N]):
        let U be an empty array
        let Q be an empty queue

        append 1 to U
        push 1 to Q

        while Q is not empty:
            pop the front element of Q into u
            for v = 1 to N:
                if A[u][v] == 1 and v is not in U:
                    append v to U
                    push v to Q

        return U
\end{verbatim}
\vspace{-3mm}

Suppose now you have an integer $N$, $M$, and an array $U$ of $N$ integers.
You wonder whether there exists a simple undirected graph with $N$ nodes and $M$ edges such that the output of the pseudocode above is the array $U$.
Construct such graph if it exists.

A simple undirected graph with $M$ edges has an adjacency matrix $A$ that satisfies the following.

\begin{itemize}
  \item $A_{u, u} = 0$ for all $1 \leq u \leq N$.
  \item Exactly $M$ pairs $(u, v)$ satisfies $1 \leq u < v \leq N$ and $A_{u, v} = 1$, meaning that there is an edge connecting node $u$ and $v$.
  \item $A_{u, v} = A_{v, u}$ for all $1 \leq u < v \leq N$.
\end{itemize}

\psection{Input}

The first line contains two integers $N$ and $M$ ($1 \le N, M \le 200\;000$).
The second line contains $N$ integers representing $U$, which is a permutation of $(1, 2, \ldots, N)$.
You are guaranteed that the first element of $U$ is always $1$.

\psection{Output}

If such a graph exists, output $M$ lines, each containing two integers $u$ and $v$ representing an edge that connects node $u$ and $v$.

If there is no such graphs, output \texttt{-1 -1} in a single line.

\exmp{1}{
5 6\\
1 5 2 3 4
}{
1 5\\
2 3\\
5 2\\
4 3\\
3 5\\
4 5
}

\textit{Explanation of Sample 1:} You can also output the following edges and get a correct answer: \\
$(1, 5), (5, 2), (2, 3), (3, 5), (2, 4), (5, 4)$.

\exmp{2}{
5 10\\
1 5 2 3 4  
}{
-1 -1
}

% \blankpage % use this if problem is only 1 page

\end{document}
